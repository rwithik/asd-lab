\documentclass[draft,10pt,a4paper,titlepage]{report}
\usepackage[utf8]{inputenc}
\usepackage{amsmath}
\usepackage{amsfonts}
\usepackage{amssymb}
\usepackage{graphicx}
\usepackage{xcolor}
\usepackage{minted}

\newcommand{\HRule}[1]{\rule{\linewidth}{#1}}

\nonstopmode


\begin{document}
{\fontfamily{cmr}\selectfont
\title{ \normalsize \textsc{}
\\ [2.0cm]
\HRule{0.5pt} \\
\LARGE \textbf{\uppercase{Quiz 1 Report}
\HRule{2pt} \\ [0.5cm]
\normalsize \today \vspace*{5\baselineskip}}
}

\date{}

\author{
	Rwithik Manoj \\
	College of Engineering, Trivandrum \\
	Department of Computer Science and Engineering }

\maketitle
\tableofcontents
\newpage

\sectionfont{\scshape}

\section{Weather Observation Station 3}

\textbf{Question:} Query a list of CITY names from STATION with even ID numbers only. You may print the results in any order, but must exclude duplicates from your answer.\newline\newline
\textbf{Solution:}\newline
\begin{verbatim}
SELECT DISTINCT city FROM Station 
	WHERE (id-2*FLOOR(id/2) = 0);
\end{verbatim}

\section{Revising the Select Query}

\textbf{Question:} Query all columns for all American cities in CITY with populations larger than 100000. The CountryCode for America is USA.\newline\newline
\textbf{Solution:}\newline
\begin{verbatim}
SELECT * FROM City 
	WHERE population > 100000 AND countrycode = 'USA';
\end{verbatim}

\section{Employee Names}

\textbf{Question:} Write a query that prints a list of employee names (i.e.: the name attribute) from the Employee table in alphabetical order.\newline\newline
\textbf{Solution:}\newline
\begin{verbatim}
SELECT name FROM Employee ORDER BY name
\end{verbatim}

\section{Average Population of Each Continent}

\textbf{Question:} Given the CITY and COUNTRY tables, query the names of all the continents (COUNTRY.Continent) and their respective average city populations (CITY.Population) rounded down to the nearest integer.\newline\newline
\textbf{Solution:}\newline
\begin{verbatim}
SELECT Country.continent, FLOOR(AVG(City.population)) 
	FROM Country, City 
	WHERE City.CountryCode = Country.code GROUP BY Country.continent;
\end{verbatim}

\section{Japan Population}

\textbf{Question:} Query the sum of the populations for all Japanese cities in CITY. The COUNTRYCODE for Japan is JPN.\newline\newline
\textbf{Solution:}\newline
\begin{verbatim}
SELECT SUM(population) FROM City WHERE countrycode = 'JPN';

\end{verbatim}

}
\end{document}
