\section{Aim}
 To study the implementation of functions, procedures and packages in PLpg/SQL

\section{{Theory}}

\subsection{Functions}

PostgreSQL functions, also known as Stored Procedures, allow you to carry out operations that would normally take several queries and round trips in a single function within the database. Functions allow database reuse as other applications can interact directly with your stored procedures instead of a middle-tier or duplicating code.
\newline
\newline

\subsubsection{Syntax}

\begin{minted}{sql}
CREATE [OR REPLACE] FUNCTION function_name (arguments) 
RETURNS return_datatype AS $variable_name$
   DECLARE
      declaration;
      [...]
   BEGIN
      < function_body >
      [...]
      RETURN { variable_name | value }
   END; LANGUAGE plpgsql;
\end{minted}

\subsection{Procedures}

A drawback of user-defined functions is that they cannot execute transactions. PostgreSQL 11 introduced stored procedures that support transactions.

\subsubsection{Syntax}

\begin{minted}{sql}
CREATE [OR REPLACE] PROCEDURE procedure_name(parameter_list)
LANGUAGE language_name
AS $$
    stored_procedure_body;
$$;
\end{minted}

\subsection{Package}

Packages are implemented as schemas, in POSTGRESQL. A schema is a named collection of tables. A schema can also contain views, indexes, sequences, data types, operators, and functions. Schemas are analogous to directories at the operating system level, except that schemas cannot be nested. PostgreSQL statement CREATE SCHEMA creates a schema.

\subsubsection{Syntax}

\begin{minted}{sql}
CREATE SCHEMA schemaname;
CREATE TABLE schemaname.tablename (
...
);
CREATE FUNCTION schemaname.functionname (
...
);
\end{minted}

\section{{Code and Output}}

}
\begin{enumerate}
\item Create a function factorial to find the factorial of a number. Use this function in a PL/pgSQL Program to display the factorial of a number.
\begin{minted}{sql}
CREATE OR REPLACE FUNCTION factorial(n INTEGER)
RETURNS INTEGER AS
$$
DECLARE
fact INTEGER := 1;
BEGIN
IF n < 1 THEN
RETURN 1;
END IF;
FOR i IN 1..n LOOP
fact := fact * i;
END LOOP;
RAISE NOTICE 'Factorial is %', fact;
RETURN fact;
END;
$$ LANGUAGE plpgsql;
\end{minted}

\newline
\includegraphics[width=\linewidth]{../Images/Functions/1.png}

\item Create a table student\_details(roll int, marks int, phone int). Create a procedure pr1 toupdate all rows in the database. Boost the marks of all students by 5\%.
\begin{minted}{sql}
CREATE OR REPLACE FUNCTION grading()
RETURNS INTEGER
AS $$
DECLARE
t INTEGER := 0
cur CURSOR FOR SELECT * FROM student
rec RECORD
grd CHAR(1)
BEGIN
OPEN cur
LOOP
t := 0
FETCH cur into rec
EXIT WHEN NOT FOUND
t := rec.m1 + rec.m2 + rec.m3
UPDATE student SET total = t WHERE CURRENT OF cur
IF t > 250 THEN
grd := 'A'
ELSIF t > 200 THEN
grd := 'B'
ELSIF t > 150 THEN
grd := 'C'
ELSIF t > 100 THEN
grd := 'D'
ELSE
grd := 'F'
END IF
CALL marks(rec.id, grd)
END LOOP
CLOSE cur
RETURN null
END
$$ LANGUAGE plpgsql

CREATE OR REPLACE PROCEDURE marks(tid INTEGER, tgrd CHAR)
LANGUAGE plpgsql
AS $$
BEGIN
UPDATE studentmark SET grade=tgrade WHERE id=tid
END
$$;
\end{minted}

\newline
\includegraphics[width=\linewidth]{../Images/Functions/2.png}

\item Create table student (id, name, m1, m2, m3, total, grade).Create a function f1 to calculate grade. Create a procedure p1 to update the total and grade.\newline
Using a function to calculate the grade.\newline
Using a procedure to update the grade value for the tuple.\newline
\begin{minted}{sql}
CREATE OR REPLACE FUNCTION grading()
RETURNS INTEGER
AS $$
DECLARE
t INTEGER := 0;
cur CURSOR FOR SELECT * FROM student;
rec RECORD;
grd CHAR(1);
BEGIN
OPEN cur;
LOOP
t := 0;
FETCH cur into rec;
EXIT WHEN NOT FOUND;
t := rec.m1 + rec.m2 + rec.m3;
UPDATE student SET total = t WHERE CURRENT OF cur;
IF t > 250 THEN
grd := 'A';
ELSIF t > 200 THEN
grd := 'B';
ELSIF t > 150 THEN
grd := 'C';
ELSIF t > 100 THEN
grd := 'D';
ELSE
grd := 'F';
END IF;
CALL marks(rec.id, grd);
END LOOP;
CLOSE cur;
RETURN null;
END;
$$ LANGUAGE plpgsql;
\end{minted}

\newline
\includegraphics[width=\linewidth]{../Images/Functions/3.png}

\item Create a package pk1 consisting of the following functions and procedures
\begin{enumerate}
\item Procedure proc1 to find the sum, average and product of two numbers
\item Procedure proc2 to find the square root of a number
\item Function named fn11 to check whether a number is even or not
\item A function named fn22 to find the sum of 3 numbers
\end{enumerate}
Use this package in a PL/SQL program. Call the functions f11, f22 and procedures pro1, pro2 within the program and display their results.

\begin{minted}{sql}

CREATE SCHEMA pk1

CREATE OR REPLACE PROCEDURE pk1.proc1(num1 REAL,num2 REAL)
LANGUAGE plpgsql
AS
$$
DECLARE
sum REAL
average REAL
prod REAL
BEGIN
sum = num1 + num2
prod = num1 * num2
average = (num1 + num2) / 2
RAISE NOTICE 'Sum of % and % is %', num1, num2, sum
RAISE NOTICE 'Product of % and % is %', num1, num2, prod
RAISE NOTICE 'Average of % and % is %', num1, num2, average
END
$$

CREATE OR REPLACE PROCEDURE pk1.proc2(num1 REAL)
LANGUAGE plpgsql
AS
$$
DECLARE
root REAL
BEGIN
root = sqrt(num1)
RAISE NOTICE 'Root of % is %', num1, root
END
$$

CREATE OR REPLACE FUNCTION pk1.fn11(num REAL) 
RETURNS VOID AS
$$
DECLARE
rem INT
BEGIN
rem = num
rem = rem %2
IF rem = 1 THEN
RAISE NOTICE 'Number % is odd', num
ELSE
RAISE NOTICE 'Number % is even', num
END IF
END
$$ LANGUAGE plpgsql

CREATE OR REPLACE FUNCTION pk1.fn22(num1 REAL, num2 REAL, num3 REAL) 
RETURNS VOID AS
$$
DECLARE
sum REAL
BEGIN
sum = num1 + num2 + num3
RAISE NOTICE 'Sum of % , %, % is %', num1, num2, num3, sum
END
$$ LANGUAGE plpgsql

CREATE OR REPLACE FUNCTION pk1.all(num1 REAL, num2 REAL, num3 REAL)
RETURNS VOID AS
$$
DECLARE
BEGIN
CALL pk1.proc1(num1, num2)
CALL pk1.proc2(num1)
PERFORM pk1.fn11(num1)
PERFORM pk1.fn22(num1, num2, num3)
END
$$ LANGUAGE plpgsql;
\end{minted}

\newline
\includegraphics[width=\linewidth]{../Images/Functions/4.png}

\end{enumerate}

\section{Result}
Implemented the program for Procedure, Functions and Packages using Postgresql 11.5 on Manjaro Linux and the output was obtained.
\newpage
\null